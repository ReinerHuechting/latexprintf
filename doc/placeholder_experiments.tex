\documentclass{scrartcl}

\usepackage{etoolbox}

\usepackage{tcolorbox}
\tcbuselibrary{listingsutf8}

\usepackage{parskip}

\pagenumbering{gobble}

% A macro for testing purposes.
\newcommand{\foo}{This is \textbackslash foo's output.}

% Expects a macro name as #1.
% Expands \#1 if it is defined, otherwise displays a message.
\newcommand{\placeholder}[1]{
    \ifcsdef{#1}
    {\csuse{#1}}
    {\textbf{#1 not defined}}
}


\begin{document}
\section*{Experiments with a placeholder macro}

\begin{tcblisting}{title={Define a macro \textbackslash foo that will be used in our experiments.}, listing only}
\newcommand{\foo}{This is \textbackslash foo's output.}
\end{tcblisting}

\begin{tcblisting}{title={Using \textbackslash ifundef from the etoolbox package.}}
\ifundef{\foo}{\textbf{foo not defined}}{\foo} \\
\ifundef{\babar}{\textbf{babar not defined}}{\babar}
\end{tcblisting}

\begin{tcblisting}{title={How to expand a macro with \textbackslash csuse.}}
\csuse{foo} \\
\csuse{babar}
\end{tcblisting}

\begin{tcblisting}{title={Combining \textbackslash ifundef and \textbackslash csuse.}}
\ifcsdef{foo}{\csuse{foo}}{\textbf{foo not defined}} \\
\ifcsdef{babar}{\csuse{babar}}{\textbf{babar not defined}}
\end{tcblisting}

\begin{tcblisting}{title={Define a macro \textbackslash placeholder.}, listing only}
\newcommand{\placeholder}[1]{
    \ifcsdef{#1}
    {\csuse{#1}}
    {\textbf{#1 not defined}}
}
\end{tcblisting}

\begin{tcblisting}{title={Using the custom \textbackslash placeholder macro.}}
\placeholder{foo} \\
\placeholder{babar}
\end{tcblisting}

\end{document}